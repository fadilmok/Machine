% !TEX TS-program = pdflatex
% !TEX encoding = UTF-8 Unicode
\documentclass[12pt,a4paper,titlepage]{article} 

\usepackage[utf8]{inputenc} % set input encoding to utf8

\makeatother

\usepackage{enumitem}
\setitemize{noitemsep,topsep=0pt,parsep=0pt,partopsep=0pt}
\usepackage{xcolor,listings}
\usepackage{textcomp}
\lstset{upquote=true}
\usepackage{graphicx}
\usepackage{subcaption}
\usepackage{hyperref}
\hypersetup{
  colorlinks=true,
  linkcolor=blue,
  urlcolor=blue
}
\usepackage{booktabs}
\usepackage{multirow}


\title{Technical Exercise 2}
\author{Fadil Mokhchane}
\date{September 6, 2017} % Delete this line to display the current date



%%% BEGIN DOCUMENT
\begin{document}

\begin{titlepage}
	\centering

	{\scshape\LARGE Fadil Mokhchane \par}
	{\small \href{https://www.linkedin.com/in/fadil/}{CV on LinkedIn} }\par
	\vspace{1cm}

	\vspace{3.5cm}
	{\huge\bfseries Technical Exercise 2\par}
	\vfill
%	{\itshape Abstract\par}
	{I was given some questions, c.f. \autoref{sec:question},
	this was my answer.}
	\vfill

% Bottom of the page
	{%Version 1\par
	\large October 16, 2017\par}
\end{titlepage}

\newpage
\tableofcontents
%\clearpage %or \cleardoublepage
\phantomsection
%\newpage
%\vfill
%
%
%\begin{table}[h!]
%  \centering
%  \caption{Table of Changes}
%%  \label{tab:table1}
%  \begin{tabular}{lll p{6.cm}}
%    \toprule
%    V. & Author & Date & Change\\
%    \midrule
%    1.0 & Fadil M. & 3/09/2017 & 
%    	Initial version. \\
%    1.1 & Fadil M. & 6/09/2017 & 
%    	Simplified and generalised the design of the code. 
%    	C.f. \autoref{apx:code}, git log and profile in the code. 
%    	
%    	- Added the Statistics engine module in \autoref{ssec:design} 
%    	
%    	- Updated implementation notes: \autoref{ssec:uniform.impl}
%    	 and \autoref{ssec:weighted.impl}, 
%    	added \autoref{ssec:stats.impl} to reflect the new design.
%    	
%    	- Updated the performances, \autoref{sec:perf}, to be clearer.
%    	
%    	- Updated the extensions, \autoref{sec:ext}.
%    	
%    	- Added \autoref{apx:code} with the code to facilitate sharing.
%    	\\
%    \bottomrule
%  \end{tabular}
%\end{table}

\newpage
\listoffigures
\listoftables
\vfill
\newpage

%-------------------------------------------------------------------------------

\section{Questions} \label{sec:question}
A machine is composed of:
\begin{itemize}
	\item an array of 32 positive integer entries (known as the *memory*).
	\item a source program of a sequence of instructions (the *program*)
	\item a single positive integer indicating the index of the next instruction to 
		be run (the *program counter*), indexed from 0
\end{itemize}

In this guide we use the notation cell(n) to refer to the entry at index n in memory

Instructions can take the following form:
\begin{itemize}
	\item `Zn`      - zeroes the memory at cell(n) , where n is an index from 0 to 31 inclusive. 
		E.g. `Z3` sets cell(3) to be equal to 0.
	\item `In`      - increments the memory at cell(n), where n is an index from 0 to 31 inclusive. 
		E.g. `I4` sets cell(4) to be equal to one greater than the previous value of cell(4).
	\item `Jn,m,t`  - if the values of cell(n) and cell(m) are **NOT** equal 
		(where n,m are both indices from 0 to 31 inclusive),
		then set the program counter to be equal to t. E.g. `J2,3,0` 
		will jump to instruction 0 if cell(2) != cell(3)
\end{itemize}

After executing any instruction, the program counter is incremented by 1, 
*unless* a J instruction succesfully branched. If the J instruction does *not* branch, 
the program counter is still incremented.

When running a program, the program counter begins at 0. At each step, 
the instruction indexed by the program counter is run, and the program counter updated appropriately. 
Execution ends when the program counter is increased past the end of the source program. 
The value in cell(0) is then returned.

\subsection{Question 1}
Write an implementation (*in the language of your choice*) of this machine.

Your program should take a list of filenames of one or more source programs to be 
run as command line arguments, e.g.

\begin{verbatim}
./machine <sourcefile1> [sourcefile2 sourcefile3 ...]
\end{verbatim}

Your program should run each program in turn, 
*without zeroing memory between executions*, and should print the result of the final program to stdout.
This allows one program to initialize memory, and the second program to then perform operations on it.

Source programs can be assumed to be ASCII text files with each instruction on a single line. 
White space should be ignored.

E.g. The below program sets cell(1) to equal 3, then copies that value into cell(0), 
increments that value, and then returns that value: 4.
\begin{verbatim}
Z1
I1
I1
I1
Z0
I0
J1,0,5
I0
\end{verbatim}

\subsection{Question 2}
Write a program that subtracts the value of cell(2) from cell(1) and stores it in cell(0). 
You may assume that cell(1) is strictly greater than cell(2).

\subsection{Question 3}
Write a program that calculates the absolute difference between cell(2) and cell(1), 
regardless of which is bigger, and stores it in celll(0).

\newpage
%-------------------------------------------------------------------------------
\section{Solution Proposed}
\subsection{Overview}

The key requirement is to deliver a production quality software.
This means that the code has to be well designed, simple, 
efficient, portable and well tested.


\subsection{Design}
\label{ssec:design}

\subsection{User Interface}


\newpage
%-------------------------------------------------------------------------------
\section{Implementation}

We use Haskell as implementation language. It is a purely functional language
statically typed.
It has a strong type system. It is all about types, they tell the story.
In this section, we introduce the key types used thoughout the application.

\subsection{General Notes}

\subsection{Analytics Module}

%-------------------------------------------------------------------------------
\section{Qualitative Analysis}
\label{sec:quality}

\subsection{Metrics}

%-------------------------------------------------------------------------------
\section{Unit and Integration Testing}
We need to test each module independtly, i.e. unit testing, and all used 
together, i.e. integration testing.

The testing environment used is described in below\footnote{
Given that I don't have access to my server, 
we run the tests on a MacBook Pro 2008 4GB DDR3 Intel Core Duo 2.53GHz.
It has 250GB harddisk, not SSD. 
We are using Haskell version 8.2.1 on MacOS. Vim version 8.0.
}.


\subsection{Test Data Sample}


\subsection{Analytics Testing} 

%-------------------------------------------------------------------------------
\section{Performance Analysis}
\label{sec:perf}

\subsection{Complexity Analysis}

\newpage
%-------------------------------------------------------------------------------
\section{Possible Extensions}
\label{sec:ext}



%-------------------------------------------------------------------------------
\newpage
\appendix
%\addcontentsline{toc}{section}{Appendices}

\section{Code} \label{apx:code}
We provide below the code. We use Vim with the extension 
\href{https://github.com/begriffs/haskell-vim-now}{haskell-vim-now}.

\end{document}
